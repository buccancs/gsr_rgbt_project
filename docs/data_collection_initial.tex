\documentclass{article}
\usepackage{amsmath,graphicx, algorithm, algorithmic}
\usepackage{geometry}
\geometry{a4paper, margin=1in}
\begin{document}

\section{Participant Handling}
\begin{itemize}
    \item \textbf{Recruitment and Consent:} Recruit adult participants with normal or corrected vision and no skin conditions. Explain the study purpose, obtain informed consent, and collect demographic information (age, gender, skin tone).
    \item \textbf{Preparation:} Allow participants to acclimate to the room for 5 minutes before starting. Clean the skin on the hands where GSR electrodes will be applied.
    \item \textbf{Electrode Attachment:} Attach the Shimmer GSR sensor electrodes to two adjacent fingertips (e.g., index and middle finger of the left hand) using conductive gel. Attach the PhysioKit GSR sensor to two fingertips of the right hand. Ensure secure contact and verify signal quality.
    \item \textbf{Seating and Comfort:} Seat the participant in front of the cameras at a fixed distance (approximately 1.0–1.5 meters). Adjust chair and camera height so the participant’s face is centred. Instruct the participant to remain as still as possible.
    \item \textbf{Instructions:} Verbally explain the task sequence and allow a brief practice if necessary. Remind participants to avoid speaking or unnecessary movements during tasks.
\end{itemize}


\section{Experimental Task Order}
The session consists of the following tasks in sequence:
\begin{itemize}
    \item \textbf{Baseline (Rest):} 5 minutes of quiet rest. The participant sits quietly, fixating on a neutral cross or blank screen to establish a physiological baseline.
    \item \textbf{Math Stressor (Cognitive Task):} 3 minutes of mentally solving difficult arithmetic problems presented on a screen. Time pressure and immediate feedback are provided to induce cognitive stress.
    \item \textbf{Inter-Task Rest:} 1 minute of quiet rest following the math task to allow partial recovery.
    \item \textbf{Relaxation (Recovery):} 5 minutes of guided deep breathing or viewing a calming nature video to induce physiological relaxation and allow GSR to return towards baseline.
    \item \textbf{Inter-Task Rest:} 1 minute of rest after the relaxation period.
    \item \textbf{Emotional Video (Affective Stimulus):} 3-minute emotionally arousing video clip (e.g., suspenseful or joyful scenes) to elicit an emotional response.
    \item \textbf{Final Rest:} 2 minutes of rest after the video to capture the recovery phase.
\end{itemize}

\begin{table}[h]
    \centering
    \caption{Task durations and inter-task intervals}
    \begin{tabular}{ll}
        \hline
        Task/Interval   & Duration  \\
        \hline
        Baseline rest   & 5 minutes \\
        Math stressor   & 3 minutes \\
        Inter-task rest & 1 minute  \\
        Relaxation      & 5 minutes \\
        Inter-task rest & 1 minute  \\
        Emotional video & 3 minutes \\
        Final rest      & 2 minutes \\
        \hline
    \end{tabular}
\end{table}


\section{Camera and Sensor Setup}
\begin{itemize}
    \item \textbf{RGB Camera:} One high-definition RGB camera (e.g., 1920x1080 at 30 fps) mounted on a tripod in front of the participant at eye level, approximately 1–1.5 meters away. The camera frame is centered on the participant’s face. Diffuse, uniform visible lighting is used to ensure clear facial illumination without glare.
    \item \textbf{Thermal Camera:} One thermal infrared camera (e.g., 640x480 at 30 fps) co-located with the RGB camera so that both cameras have a similar field of view. The camera is calibrated for human skin emissivity, and ambient temperature is controlled to avoid thermal drift.
    \item \textbf{Shimmer GSR Sensor:} Attach the Shimmer3 GSR+ sensor to two fingers (e.g., index and middle fingers) of the left hand. Sample the GSR signal at a high rate (e.g., 32–128 Hz) and stream or log data to a PC.
    \item \textbf{PhysioKit GSR Sensor:} Attach an additional GSR sensor from PhysioKit to two fingertips of the right hand. Configure the sampling rate similarly to Shimmer (e.g., 32–128 Hz). This provides a redundant GSR measurement for reliability checking.
    \item \textbf{Other Sensors (Optional):} (If applicable) Place any additional sensors (e.g., pulse oximeter, EMG) according to their guidelines, ensuring wires do not obstruct the face or interfere with the cameras.
\end{itemize}

\begin{figure}[h]
    \centering
    \includegraphics[width=0.7\textwidth]{../references/experimental_setup.png}
    \caption{Schematic of the experimental setup showing the participant seated in front of the RGB and thermal cameras (mounted on tripods), with GSR sensors attached to both hands. The left hand has the Shimmer GSR+ sensor, while the right hand has the PhysioKit GSR sensor. The cameras are positioned to capture the participant's face with uniform lighting.}
\end{figure}


\section{Synchronization Strategy}
Accurate synchronisation of video and GSR data streams is achieved by multiple methods:
\begin{itemize}
    \item \textbf{Timestamp Alignment:} Use the acquisition computer’s clock to timestamp both video frames and GSR samples. Whenever possible, connect devices to the same host or synchronise their clocks (e.g., via NTP).
    \item \textbf{Visual Marker:} At the start of recording, a bright LED flash is triggered and captured by both RGB and thermal cameras. This creates a sharp, simultaneous frame marker in the video data.
    \item \textbf{Audio/Physical Cue:} The participant or experimenter produces a hand clap or button click at the start and end of certain tasks. The clap sound (if the camera has a microphone) and any small motion artifact in the GSR create additional sync points across modalities.
    \item \textbf{Event Logging:} Use the experiment control software (or PhysioKit interface) to insert event markers (e.g., “Task Start”, “Task End”) into the data log. Manually note any irregular events or delays.
    \item \textbf{Post-Hoc Verification:} After recording, visually inspect the videos and GSR traces to confirm alignment (e.g., matching LED frame to recorded spike). Apply any constant offset correction if needed.
\end{itemize}


\section{Data Labelling and Format}
\begin{itemize}
    \item \textbf{Task Labels:} Each data file or segment is labelled with the corresponding task (e.g., \texttt{BASELINE}, \texttt{MATH}, \texttt{RELAX}, \texttt{VIDEO}). Annotate start and end times of tasks in a metadata file.
    \item \textbf{Data Files:} Save RGB and thermal video files in a lossless or high-quality format (e.g., MP4 with high bitrate, or image frame sequences). Save GSR data streams from each device as CSV or binary files with timestamps.
    \item \textbf{Metadata Logging:} Maintain a master metadata log (e.g., \texttt{session\_info.csv}) including Subject ID, session date/time, device IDs, sampling rates, and any calibration values. Record the event markers and any noted issues.
    \item \textbf{Quality Checks:} After each session, verify that data are recorded without dropouts. If any segment is missing or noisy, note this in the metadata and plan for re-recording if necessary.
\end{itemize}


\section{File Structure and Naming Convention}
\begin{itemize}
    \item \textbf{Directory Layout:} Organize data by subject. For each participant (e.g., \texttt{Subject01}), create subfolders \texttt{RGB\_Video/}, \texttt{Thermal\_Video/}, and \texttt{GSR\_Data/}.
    \item \textbf{File Naming:} Use a consistent scheme such as \texttt{SubjectID\_Task\_Modality.ext}. For example, \texttt{S01\_Baseline\_RGB.mp4}, \texttt{S01\_Math\_Thermal.mp4}, \texttt{S01\_Math\_ShimmerGSR.csv}, and \texttt{S01\_Math\_PhysioKitGSR.csv}.
    \item \textbf{Metadata Files:} Place the session log (e.g., \texttt{S01\_session.csv}) in the subject’s main folder. Include columns: Task, StartTimestamp, EndTimestamp, SensorStatus, Notes.
    \item \textbf{Backup and Versioning:} Regularly back up data to a secure storage. Use version control (e.g., Git) for any processing scripts or annotation files.
\end{itemize}

\end{document}
